I used a Reactor\+X200 arm to play go. This project has a few components. I first programmed the robot using Move\+It to pick and place go pieces in Gazebo. I then implemented a deep reinforcement learning agent to play go. Finally, I created a simple web interface to allow people to play go against the robot remotely.

\section*{Running in Robot in Gazebo}

To run the robot in Gazebo with Move\+It, run ~\newline
 
\begin{DoxyCode}
To start the node which manages the physical robot run. This will allow us to do pwm control of the gripper
       <br />
```roslaunch interbotix\_sdk arm\_run.launch robot\_name:=rx200 gripper\_operating\_mode="pwm"``` <br />
For more information on what the arm\_run.launch file does, read the interbotix\_sdk/README.md file

# ROS Packages in This Repo
go\_motion\_planning - Provides services for picking and placing go pieces using MoveIt <br />
go\_robot\_server - Web interface and server for to allow remote play with the robot <br />
interbotix\_ros\_arms - Lower level ROS nodes for ReactorX series <br />


# Docker Instructions
I had some trouble getting all the packages at the correct versions to compile so I made a Docker
       container. Here are the instructions on how to connect to the Docker container and also, if need be, create a new
       container
Put it all in a bash script?
A really good description on how to use Docker with ROS can be found
       [here](https://docs.freedomrobotics.ai/docs/ros-development-in-docker-on-mac-and-windows) <br />

## Connect to the Running Container  
1) ```sudo docker ps``` to see if the container is running

2) ```sudo docker exec -it robot\_env bash``` <br />

## Setting up the Docker Container 

1) ```sudo xhost +``` <br />
2) ```export DISPLAY=:0.0``` <br />
The first two help setup some sort of graphics dependecy within the Docker container. RVIZ won't be able to
       run without this

3) ```sudo docker pull osrf/ros:melodic-desktop-full``` <br />
The standard Docker Hub ROS images are the non-Desktop ones. These will not install neccsery graphics
       packages in order to run RVIZ

4) ```sudo docker run -it --env DISPLAY=unix$DISPLAY --privileged  --volume /tmp/.X11-unix:/tmp/.X11-unix
       -dt --name robot\_env --privileged -v /dev/ttyDXL/:/dev/ttyDXL --restart unless-stopped -v
       `pwd`:/root/workspace osrf/ros:melodic-desktop-full``` <br />

5) ```sudo docker exec -it robot\_env bash``` <br />

6) ```source ros\_entrypoint.sh``` <br />

7) ```sudo apt-get update``` <br />
Without this, rosdep won’t find any packages with the given names to install in the Docker container

8) ```mkdir /etc/udev``` <br />

9) ```sudo apt install udev``` <br />

10) ```rosdep update``` <br />

11) ```rosdep install --from-paths src --ignore-src -r -y``` <br />

12) ```sudo apt install python-pip``` <br />

13) ```sudo pip install modern\_robotics``` <br />

14) ```sudo cp catkin\_ws/src/interbotix\_ros\_arms/interbotix\_sdk/10-interbotix-udev.rules
       /etc/udev/rules.d``` <br />

15) ```sudo udevadm control --reload-rules && udevadm trigger``` <br />

16) ```sudo apt install vim``` <br />

17) ```vim usr/share/ignition/fuel\_tools/config.yam
\end{DoxyCode}
 Change the url in the config.\+yaml file to {\ttfamily \href{https://api.ignitionrobotics.org}{\tt https\+://api.\+ignitionrobotics.\+org}}

\subsection*{Restarting the Container on Reboot}

1) {\ttfamily sudo docker start robot\+\_\+env} 