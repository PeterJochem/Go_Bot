I used a Reactor\+X200 arm to play go. This project has a few components. I first programmed the robot using Move\+It to pick and place go pieces in Gazebo. I then implemented a deep reinforcement learning agent to play go. Finally, I created a simple web interface to allow people to play go against the robot remotely.

\section*{Running in Robot in Gazebo}

To run the robot in Gazebo with Move\+It, run ~\newline
 
\begin{DoxyCode}
To start the node which manages the physical robot run. This will allow us to do pwm control of the gripper
       <br />
```roslaunch interbotix\_sdk arm\_run.launch robot\_name:=rx200 gripper\_operating\_mode="pwm"``` <br />
For more information on what the arm\_run.launch file does, read the interbotix\_sdk/README.md file

# ROS Packages in This Repo
go\_motion\_planning - Provides services for picking and placing go pieces using MoveIt <br />
go\_robot\_server - Web interface and server for to allow remote play with the robot <br />
interbotix\_ros\_arms - Lower level ROS nodes for ReactorX series <br />

# Doxygen Documentation
The Doxygen documentation is available at
       file:///home/peterjochem/Desktop/Go\_Bot/catkin\_ws/src/docs/html/index.html

# Unit Testing
I used gtest to create unit tests. To run the unit test, run ```catkin\_make run\_tests``` or ```catkin\_make
       run\_tests <package\_name
\end{DoxyCode}


\section*{Docker Instructions}

I had some trouble getting all the packages at the correct versions to compile so I made a Docker container. Here are the instructions on how to connect to the Docker container and also, if need be, create a new container Put it all in a bash script? A really good description on how to use Docker with R\+OS can be found \href{https://docs.freedomrobotics.ai/docs/ros-development-in-docker-on-mac-and-windows}{\tt here} ~\newline


\subsection*{Connect to the Running Container}

1) {\ttfamily sudo docker ps} to see if the container is running

2) {\ttfamily sudo docker exec -\/it robot\+\_\+env bash} ~\newline


\subsection*{Setting up the Docker Container}

1) {\ttfamily sudo xhost +} ~\newline
 2) {\ttfamily export D\+I\+S\+P\+L\+AY=\+:0.\+0} ~\newline
 The first two help setup some sort of graphics dependecy within the Docker container. R\+V\+IZ won\textquotesingle{}t be able to run without this

3) {\ttfamily sudo docker pull osrf/ros\+:melodic-\/desktop-\/full} ~\newline
 The standard Docker Hub R\+OS images are the non-\/\+Desktop ones. These will not install neccsery graphics packages in order to run R\+V\+IZ

4) {\ttfamily sudo docker run -\/it -\/-\/env D\+I\+S\+P\+L\+AY=unix\$\+D\+I\+S\+P\+L\+AY -\/-\/privileged -\/-\/volume /tmp/.X11-\/unix\+:/tmp/.X11-\/unix -\/dt -\/-\/name robot\+\_\+env -\/-\/privileged -\/v /dev/tty\+D\+X\+L/\+:/dev/tty\+D\+XL -\/-\/restart unless-\/stopped -\/v `pwd`\+:/root/workspace osrf/ros\+:melodic-\/desktop-\/full} ~\newline


5) {\ttfamily sudo docker exec -\/it robot\+\_\+env bash} ~\newline


6) {\ttfamily source ros\+\_\+entrypoint.\+sh} ~\newline


7) {\ttfamily sudo apt-\/get update} ~\newline
 Without this, rosdep won’t find any packages with the given names to install in the Docker container

8) {\ttfamily mkdir /etc/udev} ~\newline


9) {\ttfamily sudo apt install udev} ~\newline


10) {\ttfamily rosdep update} ~\newline


11) {\ttfamily rosdep install -\/-\/from-\/paths src -\/-\/ignore-\/src -\/r -\/y} ~\newline


12) {\ttfamily sudo apt install python-\/pip} ~\newline


13) {\ttfamily sudo pip install modern\+\_\+robotics} ~\newline


14) {\ttfamily sudo cp catkin\+\_\+ws/src/interbotix\+\_\+ros\+\_\+arms/interbotix\+\_\+sdk/10-\/interbotix-\/udev.\+rules /etc/udev/rules.d} ~\newline


15) {\ttfamily sudo udevadm control -\/-\/reload-\/rules \&\& udevadm trigger} ~\newline


16) {\ttfamily sudo apt install vim} ~\newline


17) {\ttfamily vim usr/share/ignition/fuel\+\_\+tools/config.\+yaml} Change the url in the config.\+yaml file to {\ttfamily \href{https://api.ignitionrobotics.org}{\tt https\+://api.\+ignitionrobotics.\+org}}

\subsection*{Restarting the Container on Reboot}

1) {\ttfamily sudo docker start robot\+\_\+env} 